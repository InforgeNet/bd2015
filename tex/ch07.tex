\chapter{Operazioni}\label{ch:operations}
In questo Capitolo sono presentate alcune {\it operazioni interessanti}. Successivamente, nel
paragrafo~\vref{sec:operationscode}, saranno presentate le {\it implementazioni MySQL} di tali operazioni.

Le operazioni che andremo ad analizzare sono le seguenti (le stime sul numero di esecuzioni
al giorno sono basate anche su alcune delle considerazioni riportate nella {\it tavola dei volumi} del
paragrafo~\vref{sec:volumetable}):
{\tabulinesep=3pt
\begin{longtabu} to \linewidth {|>{\raggedright}X[2,m]|X[c,m]|}
\hline\rowfont\bfseries
\centering Operazione                                   & Frequenza
\\ \hline \hline \hline \hline \hline % ----------------------------------------
\endhead
1. Ottenere lo stato di una comanda.                    & \(1\,500/giorno\)
    \\ \hline % ----------------------------------------------------------------
2. Assegnamento automatico di un pony a una comanda
take-away mediante trigger.                             & \(125/giorno\)
    \\ \hline % ----------------------------------------------------------------
3. Aggiunta di una nuova comanda.                       & \(675/giorno\)
    \\ \hline % ----------------------------------------------------------------
4. Aggiunta di un nuovo piatto.                         & \(3\,375/giorno\)
    \\ \hline % ----------------------------------------------------------------
5. Controllo della disponibilità degli ingredienti in
magazzino per la produzione di una ricetta in un certo
giorno.                                                 & \(500/giorno\)
    \\ \hline % ----------------------------------------------------------------
6. Aggiunta di una prenotazione.                        & \(375/giorno\)
    \\ \hline % ----------------------------------------------------------------
7. Classifica delle recensioni.                         & \(100/giorno\)
    \\ \hline % ----------------------------------------------------------------
8. Aggiunta di una valutazione.                         & \(3/giorno\)
    \\ \hline % ----------------------------------------------------------------
\end{longtabu}}
\section{Implementazione delle operazioni}\label{sec:operationscode}
Di seguito le implementazioni MySQL delle operazioni individuate nel paragrafo precedente.

\inputminted[%
              frame=leftline,           % bordo a sinistra
              linenos,                  % attiva numeri di linea
              stepnumber=5,             % solo multipli di 5 come numeri di linea
              tabsize=4,                % dimensione del tasto tab in spazi
              fontsize=\small]%         % dimensione del font (perché stia nella pagina bisogna assicurarsi che ogni riga di codice abbia al max 80 caratteri)
                              {mysql}{./sql/operations.sql}

