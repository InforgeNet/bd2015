\section{Strutturazione dei requisiti} \label{sec:reqtables}
Le frasi delle specifiche di progetto sono state divise nelle seguenti tabelle.
Ogni tabella riporta le frasi relative ad un determinato {\it termine}. Frasi
ridondanti o superflue possono essere state rimosse.

\begin{samepage}
Nell'elencare queste frasi, si è cercato, ove possibile, di standardizzarne la
struttura in forme come le seguenti:
\begin{itemize}
    \item Per ogni \ldots, identificato da \ldots, rappresentiamo \ldots.\hfill\textit{(Attributi)}
    \item Ogni \ldots possiede \ldots.\hfill\textit{(Associazione)}
    \item I \ldots possono essere \ldots oppure \ldots.\hfill\textit{(Generalizzazione)}
    \item Se \ldots allora \ldots.\hfill\textit{(Vincolo)}
\end{itemize}
A margine, per ogni forma, è segnalato l'elemento a cui {\it solitamente}
(non necessariamente sempre) tale forma corrisponde nel diagramma concettuale.
\end{samepage}

Le frasi delle tabelle possono contenere {\it appunti e commenti}. tutto il testo fra
parentesi è da considerarsi come tale.

Inoltre alcune frasi sono state modificate (senza cambiarne il senso)
per adattarle meglio alle fasi successive della progettazione.

\begin{center}
\begin{reqtable}{DATABASE}
Il database che si intende progettare contiene i dati a supporto delle funzionalità
del sistema informativo di una catena di ristoranti. Si noti che l’oggetto
del lavoro è la progettazione di una realtà dei dati, corredata da alcune funzionalità
di back-end, implementate mediante stored procedure a livello data tier,
basato sul DBMS MySQL. In questo modo, un’applicazione distribuita può interfacciarsi
 al database, memorizzare tutti i dati che occorrono, e usufruire delle funzionalità di back-end.
Il database da progettare deve rendere possibile la gestione dei dati relativi
a tre diverse aree: area gestione, area clienti e area analytics.
\end{reqtable}
\begin{reqtable}{Area Gestione}{0.85}{20pt}
L'area gestione memorizza tutti i dati utili alla gestione dell'attività di
ristorazione, nelle sue diverse sedi.
\end{reqtable}
\begin{reqtable}{Magazzino}
Il magazzino contiene la materia prima, ossia l'insieme degli ingredienti,
organizzata in confezioni su scaffali. Un magazzino si trova presso una sede.
\end{reqtable}
\begin{reqtable}{Ingrediente}
Per ogni ingrediente, identificato dal nome, rappresentiamo la provenienza, il tipo di produzione,
il genere e se è un allergene. Ciascun ingrediente è acquistato in confezioni.
\end{reqtable}
\begin{reqtable}{Confezione}
Per ogni confezione, identificata dal codice lotto (che identifica anche produttore
e momento di produzione --- dettagli: \mbox{d.lgs.} \mbox{109/92}, art. 13 e dir. \mbox{2011/91/UE}),
rappresentiamo il peso (indicatore della quantità di prodotto), il prezzo di acquisto,
la data di acquisto, la data di arrivo nel magazzino, la data di scadenza, la
collocazione nel magazzino (sia quale magazzino che quale scaffale), l'aspetto
(se sono stati subiti danni o meno nel trasporto/stoccaggio),
lo stato (completa=in magazzino; in uso=scaricata; parziale=contenuto parzialmente svuotato)
e il contenuto residuo (se stato=parziale). Se la confezione è danneggiata allora è
possibile chiedere la sostituzione o lo sconto al venditore. Se si tiene la confezione
danneggiata allora si possono avere delle restrizioni nell'impiego degli ingredienti.
\end{reqtable}
\begin{reqtable}{Sede}
Ogni sede possiede uno o più magazzini, una cucina, uno o più tavoli divisi in una
o più sale.
\end{reqtable}
\begin{reqtable}{Cucina}
Ogni cucina possiede zero o più strumenti (macchinari o attrezzature).
\end{reqtable}
\begin{reqtable}{Strumento}
Gli strumenti possono essere macchinari o attrezzature. Per ogni strumento, identificato
dal nome, rappresentiamo le funzioni svolte e le fasi di preparazione in cui lo
strumento può essere impiegato.
\end{reqtable}
\begin{reqtable}{Pietanza}
Ogni pietanza, identificata dal menu al quale appartiene e il nome, possiede una ricetta.
\end{reqtable}
\begin{reqtable}{Ricetta}
Per ogni ricetta, identificata dal nome, rappresentiamo l'insieme di ingredienti
e il procedimento da seguire. Il procedimento è rappresentato sia sottoforma di
testo, sia da un procedimento strutturato. Il procedimento strutturato è composto da fasi.
\end{reqtable}
\begin{reqtable}{Fase}
La fase può essere una fase di
\begin{inparaenum}[a)]
\item \label{itm:add} aggiunta di ingredienti oppure di
\item \label{itm:action} manovra da compiere, che a sua volta può essere
\begin{inparaenum}[i)]
\item \label{itm:tool} con strumenti o
\item \label{itm:notool} senza strumenti.
\end{inparaenum}
\end{inparaenum}
Ogni fase è identificata dalla ricetta alla quale appartiene e da un codice (il
quale rappresenta l'ordine di esecuzione). Per (\ref{itm:add}) rappresentiamo
l'ingrediente da aggiungere e il suo dosaggio. Per (\ref{itm:action}) rappresentiamo
il testo descrittivo della manovra e il tempo necessario. Inoltre, se siamo nel
tipo (\ref{itm:tool}) rappresentiamo la lista degli strumenti necessari (in ordine
di utilizzo, dove ogni strumento ha un momento di inizio impiego e un momento di
fine impiego, relativi al tempo necessario totale della manovra --- in modo da poter
ottimizzare il parallelismo). Altimenti, se siamo nel tipo (\ref{itm:notool}) non
rappresentiamo nient'altro.
\end{reqtable}
\begin{reqtable}{Menu}
Ogni menu, identificato dalla sede e dalla data di entrata in vigore (che può essere
anche successiva alla data odierna), può essere composto solo da pietanze i cui ingredienti
sono disponibili in quantità sufficienti (stima effettuata tramite le prenotazioni
in essere e la popolarità del piatto --- se il piatto è una novità, verrà scelto
da \(\frac{1}{3}\) dei clienti stimati analizzando le prenotazioni e i clienti
presenti in passato nello stesso periodo) nei magazzini della sede oppure sono in
ordine (e arriveranno almeno 3 giorni prima della data di entrata in vigore del
menu). Il menu di una sede deve cambiare con cadenza regolare (va tenuta traccia
dei menu precedenti).
\end{reqtable}
\begin{reqtable}{Comanda}
Le comande possono essere comande da tavolo o comande take-away.
Per ogni comanda, identificata da un codice, rappresentiamo il timestamp, il
tavolo (se comanda da tavolo), l'account che ha fatto la richiesta (se comanda
take-away) l'insieme delle pietanze scelte dal menu con le eventuali
variazioni e lo stato (nuova=tutte le pietanze in attesa; in preparazione=una o più
pietanze in preparazione; parziale=una o più pietanze in servizio; evasa=tutte le
pietanze in servizio; consegna=equivalente di evasa in comande take-away). Nel
caso di comande take-away rappresentiamo anche la consegna associata. Le comande
restano sempre nel database e servono per compilare le fatture. \\
Ogni pietanza scelta ha il suo stato di avanzamento (attesa=appena inserito; in
preparazione=in fase di preparazione: deve essere possibile stimare il tempo residuo;
servizio=pronto). 
Le variazioni sono al massimo 3, scelte dallo chef nel momento in cui la
pietanza entra in menu, e possono essere
\begin{inparaenum}[a)]
\item eliminazione di ingrediente;
\item aggiunta di ingrediente oppure
\item modifica del procedimento di preparazione.
\end{inparaenum}
\end{reqtable}
\begin{reqtable}{Tavolo}
Per ogni tavolo, identificato da un codice (che identifica anche la sala a cui appartiene)
e dalla sede a cui appartiene, rappresentiamo la dimensione (numero di posti).
\end{reqtable}
\begin{reqtable}{Prenotazione}
Le prenotazioni possono essere fatte tramite account dal sito web oppure
telefondando (in questo caso la prenotazione è inserita nel database dal personale).
Inoltre possono essere prenotazioni standard o allestimenti.
Per ogni prenotazione, identificata dalla data (giorno e ora) e dal tavolo (che
può essere anche assegnato dal sistema, se la penotazione è fatta dal sito web)
o dalla sala (in caso di allestimento si prenota l'intera sala),
rappresentiamo il numero di persone (in caso di allestimento deve essere superiore
a un certo valore scelto), il recapito telefonico (solo se la prenotazione è stata
fatta telefonicamente).
Le prenotazioni possono essere rettificate fino a 48 ore prima e possono essere
annullate fino a 72 ore prima. Se il cliente che ha effettuato la prenotazione
tramite sito web non si presenta allora l'account del cliente non deve più poter
prenotare tramite sito web.
\end{reqtable}
\begin{reqtable}{Pony}
Per ogni pony, rappresentato da un codice, rappresentiamo il mezzo con il quale
si muove (2 o 4 ruote) e da uno stato attuale (libero o occupato).
\end{reqtable}
\begin{reqtable}{Consegna}
Per ogni consegna take-away, indentificata dalla comanda take-away alla quale
appartiene, rappresentiamo il pony associato, il momento in cui la merce passa nelle
mani del pony, il momento in cui la merce arriva a destinazione e il momento in
cui il pony torna alla sede.
\end{reqtable}
\begin{reqtable}{Area Clienti}{0.85}{20pt}
L’area clienti mantiene le informazioni relative all’interazione fra i clienti e la
catena di ristorazione, mediante recensioni, proposte e suggerimenti.
\end{reqtable}
\begin{reqtable}{Account}
Un account contiene informazioni di anagrafica, città di provenienza, sesso e
altri dettagli.
\end{reqtable}
\begin{reqtable}{Recensione}
Le recensioni possono essere recensioni sulle pietanze o recensioni sull'esperienza
in sede. Per ogni recensione, identificata dall'account che l'ha rilasciata e dalla
pietanza o dalla sede a cui si riferisce, rappresentiamo il giudizio globale numerico
e il testo rilasciato. Ogni recenzione possiede zero o più valutazioni degli
altri utenti (utilizzate per determinare il grado di veridicità e di accuratezza
della recensione). Ogni recensione possiede un questionario svolto, ossia una lista
di risposte a ciascuna domanda del questionario.
\end{reqtable}
\begin{reqtable}{Questionario}
Ogni questionario possiede una o più domande a risposta multipla.
\end{reqtable}
\begin{reqtable}{Domanda}
Per ogni domanda, indentificata da un codice (numero di domanda), rappresentiamo
il testo della domanda. Ogni domanda possiede due o più risposte possibili.
\end{reqtable}
\begin{reqtable}{Risposta}
Per ogni risposta, identificata dalla domanda alla quale appartiene e dal codice
(numero di risposta), rappresentiamo il testo della risposta e il punteggio
di efficienza relativo.
\end{reqtable}
\begin{reqtable}{Valutazione}
Per ogni valutazione di recensione, identificata dalla recensione alla quale si
riferisce e dall'account che ha rilasciato la valutazione, rappresentiamo il giudizio
sintetico di veridicità, il giudizio sintetico di accuratezza e il testo.
\end{reqtable}
\begin{reqtable}{Proposta}
Le proposte possono essere proposte di nuove pietanze o proposte di variazione
delle ricette già presenti (suggerimenti).
Per ogni proposta di nuova pietanza, identificata dall'account che rilascia la proposta
e da un nome, rappresentiamo il nome, gli ingredienti e il testo del procedimento
(opzionale).
Per ogni suggerimento, indentificato dall'account che rilascia il suggerimento e
da un codice, rappresentiamo la ricetta a cui si riferisce, gli ingredienti e le
fasi di procedimento da modificare. Ogni proposta possiede zero o più gradimenti
dei clienti.
\end{reqtable}
\begin{reqtable}{Gradimento}
Per ogni gradimento, identificato dalla proposta a cui si riferisce e dall'account
che ha rilasciato il gradimento, rappresentiamo il punteggio numerico.
\end{reqtable}
\begin{reqtable}{Proposta di Allestimento}
Per ogni proposta di allestimento, indentificata dall'account che ha rilasciato la
proposta e da un codice, rappresentiamo la sala della sede dove dovrebbe avvenire
l'allestimento, il nome dell'evento (scelto dal cliente), il giorno e l'ora,
il numero di partecipanti, il testo descrittivo dell'evento (disposizione di tavoli,
decorazioni, ecc\ldots) e la prenotazione relativa all'allestimento (presente solo
se la proposta è stata approvata e possibile solo se il numero dei partecipanti
è superiore ad una certa soglia).
\end{reqtable}
\end{center}
