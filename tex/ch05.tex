\chapter{Normalizzazione}
In questo Capitolo viene presentato il risultato della fase di {\it normalizzazione}.

Nel paragrafo~\vref{sec:functionaldependencies} è presentata la lista di tutte le {\it dipendenze
funzionali non banali}. Nei paragrafi ad esso successivi viene riportata la normalizzazione
di tutte le relazioni che non rispettano la {\bf Forma Normale di Boyce-Codd (BCNF)}.
% Si può notare facilmente che le uniche due tabelle a presentare
%dipendenze funzionali non banali che violano la {\bf Forma Normale di Boyce-Codd (BCNF)}
%sono:
%\begin{itemize}
%\item {\tt Sede}: {\tt Città}, {\tt Via} $\rightarrow$ {\tt CAP}
%\item {\tt Account}: {\tt Città}, {\tt Via} $\rightarrow$ {\tt CAP}
%\end{itemize}
%Queste dipendenze funzionali {\it non} sono però normalizzabili (a meno di non avere una tabella
%che metta in relazione tutte le coppie ({\tt Città}, {\tt Via}) con il relativo {\tt CAP}). Si
%noti inoltre che, per quanto riguarda la relazione {\tt Sede}, non saranno mai presenti
%due sedi con la stessa coppia ({\tt Città}, {\tt Via}) --- è infatti assurdo che una catena
%di ristorazione possieda più di una sede nella stessa via --- e quindi la {\it decomposizione}
%di tale dipendenza funzionale non porterebbe a nessun vantaggio. Per quanto riguarda la
%relazione {\tt Account}, l'introduzione dell'attributo {\tt CAP} non è richiesto dalle
%specifiche e non è necessario in quanto il {\it Codice di Avviamento Postale} è utile solo
%nel caso in cui sia necessario spedire un bene utilizzando i {\it servizi postali italiani}: le
%uniche {\it spedizioni} effettuate dalla catena di ristorazione sono le {\it consegne a
%domicilio} effettuate dai {\it pony}, che non hanno bisogno del {\it CAP} --- volendo quindi
%normalizzare anche la tabella {\tt Account}, si potrebbe semplicemente eliminare l'attributo {\tt CAP}.

Vediamo ora tutte le {\it dipendenze funzionali non banali}.
\section{Dipendenze funzionali}\label{sec:functionaldependencies}
\begin{samepage}
\begin{funcdep}{Sede}
    Città, Via $\to$ CAP\\
    Nome $\to$ Città, Via, NumeroCivico
\end{funcdep}
Poiché ({\tt Città}, {\tt Via}) non è superchiave, {\tt Sede} non
è in BCNF.\footnote{\label{note:nonbcnf}Si veda il paragrafo~\vref{sec:sedeaccount}}
\end{samepage}
\begin{funcdep}{Magazzino}
    $\emptyset$
\end{funcdep}
Poiché non ha alcuna dipendenza funzionale non banale, {\tt Magazzino} è in BCNF.
\begin{funcdep}{Confezione}
    CodiceLotto $\to$ Ingrediente, Scadenza\\
    CodiceLotto, Numero $\to$ Peso, Prezzo, DataAcquisto, DataCarico,\\
        \indent\indent\indent\indent\indent  Sede, Magazzino, Collocazione, Aspetto, Stato
\end{funcdep}
Poiché {\tt CodiceLotto} (da sé) non è superchiave, {\tt Confezione} non è in BCNF.\footnote{Si veda il paragrafo~\vref{sec:confezione}}
\begin{funcdep}{Ingrediente}
    Nome $\to$ Provienienza, TipoProduzione, Genere, Allergene
\end{funcdep}
Poiché la parte a sinistra è superchiave, {\tt Ingrediente} è in BCNF.
\begin{funcdep}{Cucina}
    Sede, Strumento $\to$ Quantità
\end{funcdep}
Poiché la parte a sinistra è superchiave, {\tt Cucina} è in BCNF.
\begin{funcdep}{Strumento}
    $\emptyset$
\end{funcdep}
Poiché non ha alcuna dipendenza funzionale non banale, {\tt Strumento} è in BCNF.
\begin{funcdep}{Funzione}
    $\emptyset$
\end{funcdep}
Poiché non ha alcuna dipendenza funzionale non banale, {\tt Funzione} è in BCNF.
\begin{funcdep}{Menu}
    ID $\to$ Sede, DataInizio, Quantità
\end{funcdep}
Poiché la parte a sinistra è superchiave, {\tt Menu} è in BCNF.
\begin{funcdep}{Elenco}
    Menu, Ricetta $\to$ Novità
\end{funcdep}
Poiché la parte a sinistra è superchiave, {\tt Elenco} è in BCNF.
\begin{funcdep}{Ricetta}
    Nome $\to$ Testo
\end{funcdep}
Poiché la parte a sinistra è superchiave, {\tt Ricetta} è in BCNF.
\begin{funcdep}{Fase}
    Ricetta, Numero $\to$ Ingrediente, Dose, Primario, Strumento, Testo,\\
        \indent\indent\indent\indent\indent Durata
\end{funcdep}
Poiché la parte a sinistra è superchiave, {\tt Ricetta} è in BCNF.
\begin{funcdep}{SequenzaFasi}
    $\emptyset$
\end{funcdep}
Poiché non ha alcuna dipendenza funzionale non banale, {\tt SequenzaFasi} è in BCNF.
\begin{funcdep}{Piatto}
    ID $\to$ Comanda, Ricetta, Stato
\end{funcdep}
Poiché la parte a sinistra è superchiave, {\tt Piatto} è in BCNF.
\begin{funcdep}{Modifica}
    $\emptyset$
\end{funcdep}
Poiché non ha alcuna dipendenza funzionale non banale, {\tt Modifica} è in BCNF.
\begin{funcdep}{Variazione}
    ID $\to$ Nome, Account
\end{funcdep}
Poiché la parte a sinistra è superchiave, {\tt Variazione} è in BCNF.
\begin{funcdep}{ModificaFase}
    ID, Variazione $\to$ Ricetta, FaseVecchia, FaseNuova
\end{funcdep}
Poiché la parte a sinistra è superchiave, {\tt ModificaFase} è in BCNF.
\begin{funcdep}{Comanda}
    ID $\to$ Timestamp, Sede, Sala, Tavolo, Account
\end{funcdep}
Poiché la parte a sinistra è superchiave, {\tt Comanda} è in BCNF.
\begin{funcdep}{Tavolo}
    ID, Sala, Sede $\to$ Posti
\end{funcdep}
Poiché la parte a sinistra è superchiave, {\tt Tavolo} è in BCNF.
\begin{funcdep}{Sala}
    $\emptyset$
\end{funcdep}
Poiché non ha alcuna dipendenza non banale, {\tt Sala} è in BCNF.
\begin{funcdep}{Consegna}
    Comanda, Pony $\to$ Sede, Partenza, Arrivo, Ritorno
\end{funcdep}
Poiché la parte a sinistra è superchiave, {\tt Consegna} non è in BCNF. 
\begin{funcdep}{Pony}
    Sede, ID $\to$ Ruote, Stato
\end{funcdep}
Poiché la parte a sinistra è superchiave, {\tt Pony} è in BCNF.
\begin{funcdep}{Prenotazione}
    ID $\to$ Sede, Data, Numero, Account, Nome, Telefono, Sala,\\
        \indent\indent\indent\indent\indent Tavolo, Descrizione, Approvato
\end{funcdep}
Poiché la parte a sinistra è superchiave, {\tt Prenotazione} è in BCNF.
\begin{funcdep}{Account}
    Città, Via $\to$ CAP\\
    Username $\to$ Email, Password, Nome, Cognome, Città, Via,\\
        \indent\indent\indent\indent\indent NumeroCivico, Telefono, PuòPrenotare
\end{funcdep}
Poiché ({\tt Città}, {\tt Via}) non è superchiave, {\tt Account} non
è in BCNF.\footnote{Si veda il paragrafo~\vref{sec:sedeaccount}.}
\begin{funcdep}{Proposta}
    ID $\to$ Account, Nome, Procedimento
\end{funcdep}
Poiché la parte a sinistra è superchiave, {\tt Proposta} è in BCNF.
\begin{funcdep}{Composizione}
    $\emptyset$
\end{funcdep}
Poiché non ha alcuna dipendenza funzionale non banale, {\tt Composizione} è in BCNF.
\begin{funcdep}{Gradimento}
    ID $\to$ Account, Proposta, Variazione, Punteggio
\end{funcdep}
Poiché la parte a sinistra è superchiave, {\tt Gradimento} è in BCNF.
\begin{funcdep}{Recensione}
    ID $\to$ Account, Ricetta, Testa, Giudizio
\end{funcdep}
Poiché la parte a sinistra è superchiave, {\tt Recensione} è in BCNF.
\begin{funcdep}{Valutazione}
    Account, Recensione $\to$ Veridicità, Accuratezza, Testo
\end{funcdep}
Poiché la parte a sinistra è superchiave, {\tt Valutazione} è in BCNF.
\begin{funcdep}{Domanda}
    Numero, Sede $\to$ Testo
\end{funcdep}
Poiché la parte a sinistra è superchiave, {\tt Domanda} è in BCNF.
\begin{funcdep}{Risposta}
    Numero, Domanda, Sede $\to$ Testo, Efficienza
\end{funcdep}
Poiché la parte a sinistra è superchiave, {\tt Risposta} è in BCNF.
\begin{funcdep}{QuestionarioSvolto}
    $\emptyset$
\end{funcdep}
Poiché non ha alcuna dipendenza funzionale non banale, {\tt QuestionarioSvolto} è in BCNF.

\section{Sede e Account}\label{sec:sedeaccount}
Le relazioni {\tt Sede} e {\tt Account} non rispettano la BCNF. A causare il problema
sono le seguenti dipendenze funzionali:
\begin{itemize}
\item {\tt Sede}: {\tt Città}, {\tt Via} $\rightarrow$ {\tt CAP}
\item {\tt Account}: {\tt Città}, {\tt Via} $\rightarrow$ {\tt CAP}
\end{itemize}
Queste dipendenze funzionali {\it non} sono però normalizzabili (a meno di non avere una tabella
che metta in relazione tutte le coppie ({\tt Città}, {\tt Via}) con il relativo {\tt CAP}). Si
noti inoltre che, per quanto riguarda la relazione {\tt Sede}, non saranno mai presenti
due sedi con la stessa coppia ({\tt Città}, {\tt Via}) --- è infatti assurdo che una catena
di ristorazione possieda più di una sede nella stessa via --- e quindi la {\it decomposizione}
di tale dipendenza funzionale non porterebbe a nessun vantaggio. Per quanto riguarda la
relazione {\tt Account}, l'introduzione dell'attributo {\tt CAP} non è richiesto dalle
specifiche e non è necessario in quanto il {\it Codice di Avviamento Postale} è utile solo
nel caso in cui sia necessario spedire un bene utilizzando i {\it servizi postali italiani}: le
uniche {\it spedizioni} effettuate dalla catena di ristorazione sono le {\it consegne a
domicilio} effettuate dai {\it pony}, che non hanno bisogno del {\it CAP} --- volendo quindi
normalizzare anche la tabella {\tt Account}, si potrebbe semplicemente eliminare
l'attributo {\tt CAP}.\footnote{In questo progetto l'attributo {\tt CAP} sarà comunque mantenuto sia per {\tt Sede} che per {\tt Account}.}

\section{Confezione}\label{sec:confezione}
La relazione {\tt Confezione} non rispetta la BCNF. Riportiamo di seguito tutte
le dipendenze funzionali non banali della relazione in questione:
\begin{funcdep}{Confezione}
    CodiceLotto $\to$ Ingrediente, Scadenza\\
    CodiceLotto, Numero $\to$ Peso, Prezzo, DataAcquisto, DataArrivo,\\
        \indent\indent\indent\indent\indent DataCarico, Sede, Magazzino, Collocazione,\\
        \indent\indent\indent\indent\indent Aspetto, Stato
\end{funcdep}

\noindent\texttt{CodiceLotto}, da sé, non è infatti {\it superchiave} della relazione {\tt Confezione}.
Per poter normalizzare tale relazione è necessario scomporla in due relazioni:

\begin{Verbatim}[commandchars=+\[\]]
CONFEZIONE(+underline[CodiceLotto], +underline[Numero], Peso, Prezzo, DataAcquisto, +textit[DataArrivo],
    +textit[DataCarico], Sede, Magazzino, +textit[Collocazione], +textit[Aspetto], Stato)
LOTTO(+underline[Codice], Ingrediente, Scadenza)
\end{Verbatim}
Dobbiamo anche aggiungere il {\it vincolo di integrità referenziale} tra {\tt CodiceLotto}
di {\tt Confezione} e {\tt Codice} di {\tt Lotto}.

\vspace{10pt}
\noindent Si vede immediatamente che, così facendo, la BCNF è rispettata.

\section{ModificaFase}\label{sec:modificafase}
La relazione {\tt Confezione} non rispetta la BCNF. Riportiamo di seguito tutte
le dipendenze funzionali non banali della relazione in questione:
\begin{funcdep}{ModificaFase}
    Variazione $\to$ Ricetta\\
    Variazione, ID $\to$ FaseVecchia, FaseNuova
\end{funcdep}
\noindent\texttt{Variazione}, da sé, non è infatti {\it superchiave} della relazione {\tt ModificaFase}.
Per poter normalizzare tale relazione è necessario spostare l'attributo {\tt Ricetta} nella
relazione {\tt Variazione}. Questo però ci impone anche di dover trovare un nuovo identificatore
per {\tt Fase} --- useremo quindi un campo {\tt ID} e toglieremo {\tt Numero} (superfluo in quanto l'ordine
delle fasi è dato dalla relazione {\tt SequenzaFasi}). Bisognerà quindi modificare le
relazioni coinvolte come segue:

\begin{Verbatim}[commandchars=+\[\]]
FASE(+underline[ID], Ricetta, +textit[Ingrediente], +textit[Dose], +textit[Primario], +textit[Strumento], +textit[Testo],
    +textit[Durata])
SEQUENZAFASI(+underline[Fase], +underline[FasePrecedente])
VARIAZIONE(+underline[ID], Ricetta, +textit[Nome], +textit[Account])
MODIFICAFASE(+underline[Variazione], +underline[ID], +textit[FaseVecchia], +textit[FaseNuova])
\end{Verbatim}
Anche i {\it vincoli di integrità referenziale} dovranno cambiare di conseguenza: dovrà
essere aggiunto il vincolo tra {\tt Ricetta} di {\tt Variazione} e {\tt Nome} di {\tt Ricetta}; dovranno
inoltre essere sistemati tutti i vincoli che si riferiscono alle tuple di {\tt Fase} in quanto
adesso vengono identificate dall'unico attributo {\tt ID}.

\vspace{10pt}
\noindent Si vede immediatamente che, così facendo, la BCNF è rispettata per tutte le relazioni modificate.

\section{QuestionarioSvolto}\label{sec:questionariosvolto}
La relazione {\tt QuestionarioSvolto} non rispetta la BCNF. Riportiamo di seguito tutte
le dipendenze funzionali non banali della relazione in questione:
\begin{funcdep}{QuestionarioSvolto}
    Recensione $\to$ Sede\\
    Recensione, Domanda $\to$ Risposta
\end{funcdep}
\noindent\texttt{Recensione}, da sé, non è infatti {\it superchiave} della relazione {\tt QuestionarioSvolto}. Nemmeno
({\tt Recensione}, {\tt Domanda}) è {\it superchiave} della relazione.
Per poter normalizzare tale relazione è necessario spostare l'attributo {\tt Sede} nella
relazione {\tt Recensione}. Questo però ci impone anche di dover trovare un nuovo identificatore
per le relazioni {\tt Risposta} e {\tt Domanda} --- useremo quindi un campo {\tt ID} per {\tt Domanda} e
toglieremo {\tt Numero} (superfluo in quanto l'ordine delle domande può essere dato dalla sequenza
degli {\tt ID}). Inoltre possiamo (anche se non necessario per la normalizzazione)
togliere {\tt Risposta} dall'identificatore di {\tt QuestionarioSvolto} (è
possibile in quanto, per ogni recensione, l'utente può dare una sola risposta ad ogni
domanda --- come specificato dalla business rule \ref{br.surveyanswers}). Bisognerà
quindi modificare le relazioni coinvolte come segue:

\begin{Verbatim}[commandchars=+\[\]]
RECENSIONE(+underline[ID], Account, Sede, Ricetta, Testo, Giudizio)
DOMANDA(+underline[ID], Sede, Testo)
RISPOSTA(+underline[Domanda], +underline[Numero], Testo, Efficienza)
QUESTIONARIOSVOLTO(+underline[Recensione], +underline[Domanda], Risposta)
\end{Verbatim}
Anche i {\it vincoli di integrità referenziale} dovranno cambiare di conseguenza: dovrà
essere aggiunto il vincolo tra {\tt Sede} di {\tt Recensione} e {\tt Nome} di {\tt Sede}; dovranno
inoltre essere sistemati il vincolo tra {\tt QuestionarioSvolto} e {\tt Risposta} in quanto le tuple di quest'ultima
adesso vengono identificate dai soli attributi {\tt Domanda} e {\tt Numero}. Ovviamente deve essere
sistemato anche il vincolo tra {\tt Risposta} e {\tt Domanda} in quanto le tuple di quest'ultima adesso
vengono identificato dall'unico attributo {\tt ID}.

\vspace{10pt}
\noindent Si vede immediatamente che, così facendo, la BCNF è rispettata per tutte le relazioni modificate.

