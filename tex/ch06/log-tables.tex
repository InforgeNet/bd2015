\section{Tabelle di Log}\label{sec:logtables}
Il seguente listato contiene il {\it codice MySQL} che implementa due tabella di log: {\tt Clienti\_Log} e {\tt Scarichi\_Log}.

La prima tabella mantiene, per ogni sede e per ogni mese, il numero di clienti che si sono presentati
{\it senza prenotazione}. Tale informazione sarà poi utilizzata per stimare il numero di
clienti presenti in sala in un dato giorno al fine di calcolare la quantità di ingredienti
necessari per produrre le ricette del menu.

Non è necessario registrare il numero di clienti che si sono
presentati con prenotazione in quanto tale informazione può essere facilmente ricavata
dalla tabella {\tt Prenotazione}.

L'attributo contatore {\tt SenzaPrenotazione} deve essere incrementato manualmente
dallo Staff del ristorante ogni qual volta che si presenta un cliente senza prenotazione. Per
farlo, è sufficiente chiamare la {\it stored procedure} {\tt RegistraClienti(VARCHAR(45), INT)}.

La seconda tabella mantiene invece le informazioni su tutti gli scarichi effettuati dai magazzini
per ogni ingrediente. Tale informazione sarà poi utilizzata per effettuare l'analisi dei
consumi e degli sprechi.

\inputminted[%
              frame=leftline,           % bordo a sinistra
              linenos,                  % attiva numeri di linea
              stepnumber=5,             % solo multipli di 5 come numeri di linea
              tabsize=4,                % dimensione del tasto tab in spazi
              fontsize=\small]%         % dimensione del font (perché stia nella pagina bisogna assicurarsi che ogni riga di codice abbia al max 80 caratteri)
                              {mysql}{./sql/lt.sql}
