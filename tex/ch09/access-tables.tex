\section{Nuove tavole degli accessi}\label{sec:newaccesstables}
Vediamo come cambiano le tavole degli accessi per alcune operazioni in seguito all'introduzione
delle ridondanze individuate in questo Capitolo.
\vspace{10pt}

La prima ridondanza impatta sulle operazioni 5 e 6 (la seconda esegue il trigger che aggiorna la ridondanza):
{\tabulinesep=3pt
\begin{longtabu} to \linewidth {|X[2,c,m]|X[c,m]|X[c,m]|}
\hline\rowfont\bfseries
\multicolumn{3}{|c|}{\large Operazione 5}
\\\hline\hline\hline\hline
\textbf{Tabella}                        & \textbf{Accessi}      & \textbf{Tipo}
\\ \hline \hline \hline % ------------------------------------------------------
\endhead
MV\_ClientiPrenotazione                 & \(1\)                 & L
    \\ \hline % ----------------------------------------------------------------
Clienti\_Log                            & \(5\)                 & L
    \\ \hline % ----------------------------------------------------------------
MV\_OrdiniRicetta                       & \(1\)                 & L
    \\ \hline % ----------------------------------------------------------------
Fase                                    & \(7\)                 & L
    \\ \hline % ----------------------------------------------------------------
Confezione                              & \(2\)                 & L
    \\ \hline\hline\hline % ----------------------------------------------------
\multicolumn{3}{|l|}{\textbf{Costo totale:} \(1 + 5 + 1 + 7 + 2 = 16 \times 500 = \textbf{8\,000}/giorno\)}
    \\ \hline % ----------------------------------------------------------------
\end{longtabu}}

{\tabulinesep=3pt
\begin{longtabu} to \linewidth {|X[2,c,m]|X[c,m]|X[c,m]|}
\hline\rowfont\bfseries
\multicolumn{3}{|c|}{\large Operazione 6}
\\\hline\hline\hline\hline
\textbf{Tabella}                        & \textbf{Accessi}      & \textbf{Tipo}
\\ \hline \hline \hline % ------------------------------------------------------
\endhead
Prenotazione                            & \(1\)                 & S
    \\ \hline % ----------------------------------------------------------------
MV\_ClientiPrenotazione                 & \(1\)                 & S
    \\ \hline\hline\hline % ----------------------------------------------------
\multicolumn{3}{|l|}{\textbf{Costo totale:} \(1 \times 2 + 1 \times 2 = 4 \times 375 = \textbf{1\,500}/giorno\)}
    \\ \hline % ----------------------------------------------------------------
\end{longtabu}}

\noindent In definitiva:\\
\textit{Costo senza ridondanza:} \(15\,000 + 750 = \textbf{15\,750}/giorno\)\\
\textit{Costo con ridondanza:} \(8\,000 + 1\,500 = \textbf{9\,500}/giorno\)

Gli altri due trigger usati per tenere la ridondanza aggiornata possono essere trascurati
in quanto vengono eseguiti molto raramente. La ridondanza risulta conveniente: decidiamo quindi di mantenerla.
\vspace{10pt}

La seconda ridondanza impatta sulle operazioni 7 e 8 (di nuovo, la seconda esegue il trigger che aggiorna la ridondanza):
{\tabulinesep=3pt
\begin{longtabu} to \linewidth {|X[2,c,m]|X[c,m]|X[c,m]|}
\hline\rowfont\bfseries
\multicolumn{3}{|c|}{\large Operazione 7}
\\\hline\hline\hline\hline
\textbf{Tabella}                        & \textbf{Accessi}      & \textbf{Tipo}
\\ \hline \hline \hline % ------------------------------------------------------
\endhead
Recensione                              & \(2\,000\)            & L
    \\ \hline\hline\hline % ----------------------------------------------------
\multicolumn{3}{|l|}{\textbf{Costo totale:} \(2\,000 \times 100 = \textbf{20\,000}/giorno\)}
    \\ \hline % ----------------------------------------------------------------
\end{longtabu}}

\clearpage
{\tabulinesep=3pt
\begin{longtabu} to \linewidth {|X[2,c,m]|X[c,m]|X[c,m]|}
\hline\rowfont\bfseries
\multicolumn{3}{|c|}{\large Operazione 8}
\\\hline\hline\hline\hline
\textbf{Tabella}                        & \textbf{Accessi}      & \textbf{Tipo}
\\ \hline \hline \hline % ------------------------------------------------------
\endhead
Valutazione                             & \(1\)                 & S
    \\ \hline % ----------------------------------------------------------------
Recensione                              & \(1\)                 & S
    \\ \hline\hline\hline % ----------------------------------------------------
\multicolumn{3}{|l|}{\textbf{Costo totale:} \(1 \times 2 + 1 \times 2 = 4 \times 3 = \textbf{12}/giorno\)}
    \\ \hline % ----------------------------------------------------------------
\end{longtabu}}

\noindent In definitiva:\\
\textit{Costo senza ridondanza:} \(60\,000 + 6 = \textbf{60\,006}/giorno\)\\
\textit{Costo con ridondanza:} \(20\,000 + 12 = \textbf{20\,012}/giorno\)

La ridondanza risulta conveniente: decidiamo quindi di mantenerla.
