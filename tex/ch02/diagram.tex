\section{Diagramma entità-relazione}
il {\it diagramma concettuale} finale è mostrato \vpageref{fig:conceptdiagram}.

Si noti che si è deciso di aggiungere una {\it ridondanza} nell'entità {\tt Comanda}:
l'attributo {\tt Stato} della comanda può infatti essere determinato dallo stato di
tutti i piatti ordinati nella comanda. Decidiamo però di mantenere tale ridondanza:
il motivo di questa decisione sarà discusso nel \vref{ch:redundancies}.

La spiegazione di come sono stati realizzati il {\it procedimento strutturato} e le {\it variazioni}
(e quindi anche i {\it suggerimenti}) è disponibile nel paragrafo \vref{sec:structuredprocess}.
\clearpage
\begin{landscape}
\thispagestyle{empty}
\label{fig:conceptdiagram}
\end{landscape}
