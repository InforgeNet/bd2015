\subsection{Relazioni}
Nella seguente tabella sono elencate e descritte tutte le {\it relazioni} (o {\it associazioni}), inserite
nell'ordine in cui sono state individuate.
{\tabulinesep=3pt
\begin{longtabu} to \linewidth {|X[c,m]|X[c,m]|X[c,m]|>{\raggedright}X[2,m]|}
\hline\rowfont\bfseries
Relazione   & \specialcell{Entità A\\(Cardinalità)}
                            & \specialcell{Entità B\\(Cardinalità)}
                                            & \centering Descrizione
\\ \hline \hline \hline \hline \hline % ----------------------------------------
\endhead
Disponibilità
            & \specialcell{Sede\\(1,N)}
                            & \specialcell{Magazzino\\(1,1)}
& Una sede può avere uno o più magazzini. Un magazzino appartiene solo a una sede.
    \\ \hline % ----------------------------------------------------------------
Stoccaggio  & \specialcell{Magazzino\\(0,N)}
                            & \specialcell{Confezione\\(0,1)}
& Le confezioni sono mantenute all'interno dei magazzini. Una confezione può anche essere in ordine (e quindi non in stoccaggio).
    \\ \hline % ----------------------------------------------------------------
InOrdine
            & \specialcell{Magazzino\\(0,N)}
                            & \specialcell{Confezione\\(0,1)}
& Una confezione può essere in ordine o in stoccaggio ad un solo magazzino. Un magazzino può avere in ordine molte confezioni.
    \\ \hline % ----------------------------------------------------------------
Contenuto   & \specialcell{Confezione\\(1,1)}
                            & \specialcell{Ingrediente\\(0,N)}
& Una confezione contiene un solo ingrediente. Un certo ingrediente può essere contenuto in più confezioni (zero nel caso di ingrediente esaurito).
    \\ \hline % ----------------------------------------------------------------
Cucina
            & \specialcell{Sede\\(0,N)}
                            & \specialcell{Strumento\\(0,N)}
& In una cucina della sede ci possono essere più tipologie di strumenti. Un tipo di strumento può essere utilizzato in più sedi.
    \\ \hline % ----------------------------------------------------------------
Applicazione
            & \specialcell{Sede\\(1,N)}
                            & \specialcell{Menu\\(1,1)}
& Una sede applica un menu alla volta. I menu passati vengono comunque mantenuti ma un menu appartiene solo ad una sede.
    \\ \hline % ----------------------------------------------------------------
Elenco      & \specialcell{Menu\\(1,N)}
                            & \specialcell{Ricetta\\(0,N)}
& Una ricetta appartiene a zero o più menu e un menu possiede più ricette (almeno una).
    \\ \hline % ----------------------------------------------------------------
Procedimento
            & \specialcell{Ricetta\\(1,N)}
                            & \specialcell{Fase\\(1,1)}
& Ogni ricetta è formata da almeno una fase. Ogni fase appartiene a una sola ricetta.
    \\ \hline % ----------------------------------------------------------------
Precedente
            & \specialcell{Fase\\(0,N)}
                            & \specialcell{Fase\\(0,N)}
& Questa relazione permette di rappresentare la sequenza delle fasi. Una fase può precedere zero o più fasi e avere zero o più fasi precedenti.
    \\ \hline % ----------------------------------------------------------------
Aggiunta
            & \specialcell{Fase\\(1,1)}
                            & \specialcell{Ingrediente\\(0,N)}
& Un ingrediente può essere usato da zero o più fasi. Una fase aggiunge al massimo un ingrediente.
    \\ \hline % ----------------------------------------------------------------
Utilizzo
            & \specialcell{Fase\\(0,1)}
                            & \specialcell{Strumento\\(0,N)}
& Uno strumento può essere utilizzato in zero o più fasi. Una fase può utilizzare al massimo uno strumento.
    \\ \hline % ----------------------------------------------------------------
Produzione  & \specialcell{Ricetta\\(0,N)}
                            & \specialcell{Piatto\\(1,1)}
& Un piatto è prodotto secondo una sola ricetta. Ogni ricetta può essere scelta anche più volte da tanti clienti (e per ogni scelta viene prodotto un piatto).
    \\ \hline % ----------------------------------------------------------------
Modifica
            & \specialcell{Piatto\\(0,N)}
                            & \specialcell{VariazionePiatto\\(0,N)}
& Un piatto può applicare da zero a tre variazioni. Una variazione può essere scelta in più piatti (della stessa ricetta).
    \\ \hline % ----------------------------------------------------------------
Procedura
            & \specialcell{Variazione\\(1,N)}
                            & \specialcell{ModificaFase\\(1,1)}
& Una variazione prevede almeno la modifica di una fase. Ogni modifica di fase appartiene a una sola variazione.
    \\ \hline % ----------------------------------------------------------------
Nuova
            & \specialcell{AggiuntaFase\\(1,1)}
                            & \specialcell{Fase\\(0,N)}
& Una aggiunta di fase può aggiungere una fase. Una fase può essere aggiunta da più aggiunte di fasi.
    \\ \hline % ----------------------------------------------------------------
Vecchia
            & \specialcell{EliminazioneFase\\(1,1)}
                            & \specialcell{Fase\\(0,N)}
& Una eliminazione di fase può rimuovere una fase. Una fase può essere rimossa da più eliminazioni di fasi.
    \\ \hline % ----------------------------------------------------------------
Nuova
            & \specialcell{SostituzioneFase\\(1,1)}
                            & \specialcell{Fase\\(0,N)}
& Una sostituzione di fase deve aggiungere una fase. Una fase può essere aggiunta in più sostituzioni di fasi.
    \\ \hline % ----------------------------------------------------------------
Vecchia
            & \specialcell{SostituzioneFase\\(1,1)}
                            & \specialcell{Fase\\(0,N)}
& Una sostituzione di fase può rimuovere una fase. Una fase può essere rimossa in più sostituzioni di fasi.
    \\ \hline % ----------------------------------------------------------------
Ordine
            & \specialcell{Comanda\\(1,N)}
                            & \specialcell{Piatto\\(1,1)}
& Una comanda può ordinare uno o più piatti. Ogni piatto è ordinato da una sola comanda. 
    \\ \hline % ----------------------------------------------------------------
Gestione
            & \specialcell{Sede\\(0,N)}
                            & \specialcell{Comanda\\(1,1)}
& Una sede può aver ricevuto zero o più comande. Una comanda è inviata a una sola sede.
    \\ \hline % ----------------------------------------------------------------
Suddivisione
            & \specialcell{Sede\\(1,N)}
                            & \specialcell{Sala\\(1,1)}
& Una sede è suddivisa in più sale, minimo una. Ogni sala appartiente a una Sede.
    \\ \hline % ----------------------------------------------------------------
Disposizione
            & \specialcell{Sala\\(1,N)}
                            & \specialcell{Tavolo\\(1,1)}
& Una sala può contenere uno o più tavoli. Ogni tavolo è sempre situato in una sala.
    \\ \hline % ----------------------------------------------------------------
Mittente
            & \specialcell{Tavolo\\(0,N)}
                            & \specialcell{ComandaTavolo\\(1,1)}
& Un tavolo può aver inviato zero o più comande. Una comanda è inviata da un solo tavolo.
    \\ \hline % ----------------------------------------------------------------
Richiesta
            & \specialcell{ComandaTakeAway\\(0,1)}
                            & \specialcell{Consegna\\(1,1)}
& Ogni consegna corrisponde a una sola comanda. A una comanda viene associata una consegna solo quando è pronta.
    \\ \hline % ----------------------------------------------------------------
Flotta
            & \specialcell{Sede\\(0,N)}
                            & \specialcell{Pony\\(1,1)}
& Un pony lavora presso una sola sede. Ogni flotta è composta anche da più pony.
    \\ \hline % ----------------------------------------------------------------
Trasporto
            & \specialcell{Consegna\\(1,1)}
                            & \specialcell{Pony\\(0,N)}
& Un pony può aver effettuato zero o più consegne. Ogni consegna è gestita da un solo pony.
    \\ \hline % ----------------------------------------------------------------
Riserva
            & \specialcell{Prenotazione\\(0,1)}
                            & \specialcell{Tavolo\\(0,N)}
& Un tavolo può essere stato riservato da più prenotazioni (in momenti diversi). Una prenotazione può riservare solo un tavolo.
    \\ \hline % ----------------------------------------------------------------
InvioPre
            & \specialcell{PrenotazioneOnline\\(1,1)}
                            & \specialcell{Account\\(0,N)}
& Con un account è possibile effettuare prenotazioni online. Una prenotazione può essere effettuata da un solo account.
    \\ \hline % ----------------------------------------------------------------
Ordinazione
            & \specialcell{Account\\(0,N)}
                            & \specialcell{ComandaTakeAway\\(1,1)}
& Da un account è possibile effettuare ordinazioni. Ogni comanda take-away è ordinata da un solo account.
    \\ \hline % ----------------------------------------------------------------
InvioPro
            & \specialcell{Account\\(0,N)}
                            & \specialcell{Proposta\\(1,1)}
& Con un account è possibile inviare delle proposte. Ogni proposta è inviata da un solo account.
    \\ \hline % ----------------------------------------------------------------
Composizione
            & \specialcell{Proposta\\(1,N)}
                            & \specialcell{Ingrediente\\(0,N)}
& Una proposta è composta da almeno un ingrediente. Un ingrediente può essere presente in zero o più proposte.
    \\ \hline % ----------------------------------------------------------------
InvioSug
            & \specialcell{Account\\(0,N)}
                            & \specialcell{Suggerimento\\(1,1)}
& Un cliente che ha un account può rilasciare suggerimenti per le ricette. Un suggerimento può essere inviato da un solo account.
    \\ \hline % ----------------------------------------------------------------
InvioGra
            & \specialcell{Account\\(0,N)}
                            & \specialcell{Gradimento\\(1,1)}
& Con un account è possibile inviare zero o più gradimenti. Ogni gradimento è inviato da un solo account.
    \\ \hline % ----------------------------------------------------------------
RiferimentoP 
            & \specialcell{GradimentoProposta\\(1,1)}
                            & \specialcell{Proposta\\(0,N)}
& Una proposta può essere valutata da zero o più gradimenti. Ogni gradimento si riferisce solo a una proposta.
    \\ \hline % ----------------------------------------------------------------
RiferimentoS
            & \specialcell{GradimentoSuggerimento\\(1,1)}
                            & \specialcell{Suggerimento\\(0,N)}
& Un gradimento valuta un solo suggerimento. Una suggerimento può essere valutato da zero o più gradimenti.
    \\ \hline % ----------------------------------------------------------------
Rilascio
            & \specialcell{Account\\(0,N)}
                            & \specialcell{Recensione\\(1,1)}
& Con un account è possibile rilasciare più recensioni. Ogni recensione è rilasciata da un solo account.
    \\ \hline % ----------------------------------------------------------------
InvioVal
            & \specialcell{Account\\(0,N)}
                            & \specialcell{Valutazione\\(1,1)}
& Con un account è possibile inviare valutazioni. Ogni valutazione è rilasciata da un solo account.
    \\ \hline % ----------------------------------------------------------------
RelativaA
            & \specialcell{Valutazione\\(1,1)}
                            & \specialcell{Recensione\\(0,N)}
& Una recensione può avere nessuna o molte valutazioni. Ogni valutazione valuta una sola recensione.
    \\ \hline % ----------------------------------------------------------------
Consiglio
            & \specialcell{Recensione\\(1,1)}
                            & \specialcell{Ricetta\\(0,N)}
& Ogni recensione si riferisce a una sola ricetta. Una ricetta può essere stata recensita zero o più volte.
    \\ \hline % ----------------------------------------------------------------
Relazione
            & \specialcell{Sede\\(1,1)}
                            & \specialcell{Questionario\\(1,1)}
& Ogni sede ha uno e un solo questionario. Ogni questionario appartiene a una sola sede.
    \\ \hline % ----------------------------------------------------------------
Quiz
            & \specialcell{Questionario\\(1,N)}
                            & \specialcell{Domanda\\(1,1)}
& Un questionario è composto da una o più domande. Una domanda appartiene a un solo questionario.
    \\ \hline % ----------------------------------------------------------------
Opzione
            & \specialcell{Domanda\\(1,N)}
                            & \specialcell{Risposta\\(1,1)}
& Una domanda può avere una o più risposte (di solito almeno due). Ogni risposta si riferisce a una sola domanda.
    \\ \hline % ----------------------------------------------------------------
IstanzaDi
            & \specialcell{Questionario\\(0,N)}
                            & \specialcell{QuestionarioSvolto\\(1,1)}
& Un questionario svolto è istanza di un solo questionario. Un questionario può essere svolto da diverse persone in diverse recensioni.
    \\ \hline % ----------------------------------------------------------------
Compilazione
            & \specialcell{QuestionarioSvolto\\(1,N)}
                            & \specialcell{Risposta\\(0,1)}
& Un questionario svolto contiene le risposte alle domande. Una risposta può essere stata data in zero o più questionari svolti.
    \\ \hline % ----------------------------------------------------------------
Appartenenza
            & \specialcell{QuestionarioSvolto\\(1,1)}
                            & \specialcell{Recensione\\(1,1)}
& Un questionario svolto appartiene a una e una sola recensione. Una recensione ha un solo questionario svolto.
    \\ \hline % ----------------------------------------------------------------
SerataTema
            & \specialcell{Allestimento\\(1,1)}
                            & \specialcell{Sala\\(0,N)}
& Un'allestimento prenota una sola sala per una serata a tema. In una sala possono essere state svolte zero o più serate a tema.
    \\ \hline % ----------------------------------------------------------------
\end{longtabu} }
