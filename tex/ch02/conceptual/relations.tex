\subsection{Relazioni}
Nella seguente tabella sono elencate e descritte tutte le {\it relazioni} (o {\it associazioni}), inserite
nell'ordine in cui sono state individuate.
{\tabulinesep=3pt
\begin{longtabu} to \linewidth {|X[c,m]|X[c,m]|X[c,m]|>{\raggedright}X[2,m]|}
\hline\rowfont\bfseries
Relazione   & \specialcell{Entità A\\(Cardinalità)}
                            & \specialcell{Entità B\\(Cardinalità)}
                                            & \centering Descrizione
\\ \hline \hline \hline \hline \hline % ----------------------------------------
\endhead
Disponibilità
            & \specialcell{Sede\\(1,N)}
                            & \specialcell{Magazzino\\(1,1)}
                                            & Una sede possiede uno o più magazzini.
                                              Un magazzino appartiene a una e una
                                              sola sede.
    \\ \hline % ----------------------------------------------------------------
Stoccaggio  & \specialcell{Magazzino\\(0,N)}
                            & \specialcell{Confezione\\(0,1)}
                                            & Le confezioni sono mantenute
                                              all'interno dei magazzini. Una confezione,
                                              rispetto a magazzino, può anche essere in ordine
                                              (e quindi non in stoccaggio).
    \\ \hline % ----------------------------------------------------------------
InOrdine    & \specialcell{Magazzino\\(0,N)}
                            & \specialcell{Confezione\\(0,1)}
                                            & Se una confezione non è in stoccaggio
                                              allora è in ordine. In questo caso
                                              non dovrà avere alcuni attributi (Collocazione,
                                              Aspetto, ecc.)
    \\ \hline % ----------------------------------------------------------------
Contenuto   & \specialcell{Confezione\\(1,1)}
                            & \specialcell{Ingrediente\\(0,N)}
                                            & Una confezione contiene uno e un solo
                                              tipo di ingrediente. Un certo ingrediente
                                              può essere contenuto in più confezioni (anche
                                              zero nel caso un ingrediente sia stato
                                              esaurito).
    \\ \hline % ----------------------------------------------------------------
Cucina      & \specialcell{Sede\\(0,N)}
                            & \specialcell{Strumento\\(1,N)}
                                            & La cucina mette in relazione gli strumenti
                                              con la sede alla quale appartengono.
                                              Dovrà riportare anche la quantità di strumenti
                                              di un certo tipo appartenenti alla sede.
    \\ \hline % ----------------------------------------------------------------
Applicazione
            & \specialcell{Sede\\(1,N)}
                            & \specialcell{Menu\\(1,1)}
                                            & Una sede applica un menu alla volta.
                                              I menu passati vengono comunque mantenuti
                                              ma un menu appartiene solo ad una sede
                                              (nel caso di menu uguali in sedi diversi
                                              saranno rappresentati come 2 menu diversi).
    \\ \hline % ----------------------------------------------------------------
Elenco      & \specialcell{Menu\\(1,N)}
                            & \specialcell{Ricetta\\(0,N)}
                                            & Una ricetta appartiene a zero o più menu
                                              e un menu possiede più ricette (almeno una).
    \\ \hline % ----------------------------------------------------------------
% ...FASE...
Produzione  & \specialcell{Ricetta\\(0,N)}
                            & \specialcell{Piatto\\(1,1)}
                                            & Un piatto è prodotto secondo una e una sola
                                              ricetta. Ogni ricetta può essere scelta
                                              anche più volte da tanti clienti (e per ogni
                                              scelta viene prodotto un piatto).
    \\ \hline % ----------------------------------------------------------------
\ldots      & \specialcell{\ldots\\(\ldots)}
                            & \specialcell{\ldots\\(\ldots)}
                                            & \ldots
    \\ \hline % ----------------------------------------------------------------
% ...VARIAZIONE...
% ... continuare
\end{longtabu} }
