\subsection{Generalizzazioni}
\begin{itemize}
\item Le due entità {\tt Attrezzatura} e {\tt Macchinario} sono entrambi
    strumenti da cucina. Condividono molte caratteristiche e per tal motivo è
    utile introdurre un'entità padre: {\tt Strumento}. La generalizzazione è
    {\it totale}, in quanto in cucina si possono avere solo attrezzature o
    macchinari, ed {\it esclusiva}.
\item Ogni {\tt Menu} che è stato sostituito da uno nuovo deve essere comunque
    mantenuto nel database. Per tale motivo è utile aggiungere l'entità
    {\tt MenuPassato} che rappresenta un menu non più in uso. In questo modo
    {\tt MenuPassato} è un sottoinsieme di {\tt Menu} e quindi è una generalizzazione
    {\it parziale}.
\item La {\tt Fase} \ldots                                                             % TODO: FASE
\item La {\tt Comanda} può essere di due tipi: {\tt ComandaTavolo} oppure
    {\tt ComandaTakeAway}. La generalizzazione è {\it totale} ed
    {\it esclusiva}.
\item La {\tt Prenotazione} può essere di due tipi: una {\tt PrenotazioneOnline} (dal sito)
    oppure una {\tt PrenotazioneTelefonica}. La generalizzazione è {\it totale}
    ed {\it esclusiva}. Una {\tt PrenotazioneOnline} a sua volta può essere
    una semplice {\tt PrenotazioneOnline} oppure un {\tt Allestimento} (entità già individuata).
    In questo caso la generalizzazione è {\it parziale} e {\tt Allestimento}
    è un sottoinsieme di {\tt PrenotazioneOnline}.
\item Un {\tt Gradimento} può riferirsi a un {\tt Suggerimento} o a una {\tt Proposta}.
    Risulta quindi comodo aggiungere le entità {\tt GradimentoSuggerimento} e
    {\tt GradimentoProposta} come figlie. La generalizzazione è {\it totale} ed
    {\it esclusiva}.
\end{itemize}

Devono essere quindi aggiunte le seguenti entità:
\begin{multicols}{2}
\begin{itemize}
    \item\tt Strumento
    \item\tt MenuPassato                                                               % TODO: FASE
    \item\tt ComandaTavolo
    \item\tt ComandaTakeAway
    \item\tt PrenotazioneOnline
    \item\tt PrenotazioneTelefonica
    \item\tt GradimentoSuggerimento
    \item\tt GradimentoProposta
\end{itemize}
\end{multicols}
