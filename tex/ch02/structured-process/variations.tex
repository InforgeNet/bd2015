\subsection{Variazioni}
Con la struttura mostrata nel paragrafo \vref{sec:structuredprocess} è possibile anche
gestire le {\it variazioni}.

Le variazioni devono poter: aggiungere nuove fasi; rimuovere fasi.

Prendiamo ad esempio la seguente struttura che rappresenta il procedimento di una
pizza al prosciutto:

\vspace{5pt}\centerline{\includegraphics[width=0.8\textwidth]{ex-struct-process-2}}

\vspace{15pt}

Dove:
\begin{enumerate}[label=$F_{\arabic*}$:]
    \item fase di stesura della pasta;
    \item fase di aggiunta della pasta al piatto;
    \item fase di produzione del sugo di pomodoro;
    \item fase di aggiunta del pomodoro lavorato (sugo);
    \item fase di aggiunta della mozzarella;
    \item fase di taglio del prosciutto in fette;
    \item fase di aggiunta del prosciutto lavorato (fette di prosciutto) al piatto;
    \item fase di cottura.
\end{enumerate}
Si noti che con tale struttura ci sono più modi per rappresentare lo stesso procedimento
di preparazione di un piatto: ad esempio si potrebbe scambiare $F_{1}$ con $F_{2}$ senza
cambiare il {\it senso} del procedimento.

Facciamo l'ipotesi che il cliente voglia come variazione la rimozione del pomodoro
(pizza bianca). In tal caso devono essere rimosse due fasi (anche se la variazione è
una sola!): $F_{3}$ e $F_{4}$. Se invece quello precedente fosse stato il procedimento
di una pizza margherita e il cliente avesse indicato come variazione l'aggiunta del
prosciutto, si sarebbero dovute aggiungere due fasi: $F_{6}$ e $F_{7}$.

Questo dimostra che è necessario che ogni variazione possa effettuare anche più di un'operazione
elementare (aggiunta o rimozione) sul procedimento strutturato.

L'entità {\tt ModificaFase} (con le sue entità figlie) è stata aggiunta al {\it diagramma
entità-relazione} (disponibile a \vpageref{diagram.1}) allo scopo di permettere
ad ogni variazione di {\it modificare} anche più fasi: ogni variazione ha più
{\tt ModificaFase} e ogni {\tt ModificaFase} rappresenta la modifica (aggiunta, eliminazione, sostituzione)
di una e una sola fase (la sostituzione coinvolge però due fasi: quella da togliere e quella da aggiungere).

Le fasi che fanno parte di variazioni (come le fasi $F_{6}$ e $F_{7}$ nel caso del procedimento
della pizza margherita) non vengono confuse dal database con le fasi della ricetta {\it normale}:
infatti le fasi che fanno parte di variazioni si troveranno come {\it chiavi esterne}
all'interno dell'entità {\tt AggiuntaFase} (o {\tt SostituzioneFase}) per mezzo della relazione
{\tt Nuova}, e quindi il database saprà distinguerle da quelle della ricetta normale
(vedere il diagramma \vpageref{diagram.1}).
