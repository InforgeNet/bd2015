\chapter{Implementazione MySQL}\label{ch:sqlcode}
In questo Capitolo viene presentato lo {\it script MySQL che implementa il database}.

Si noti che sono stati aggiunti alcuni {\it trigger} (rispetto a quelli presentati nel
paragrafo~\vref{sec:triggers}) e altri sono stati modificati. Sono
stati anche aggiunti i codici per {\it simulare} il comportamento dell'{\tt AUTO\_INCREMENT}. Infatti
l'{\it engine} {\tt InnoDB} non permette di avere l'{\tt AUTO\_INCREMENT} sull'attributo {\tt ID} in tabelle
come {\tt Magazzino}. Infatti in {\tt Magazzino} le tuple dovrebbero assumere la seguente forma:
\begin{Verbatim}
$ SELECT * FROM Magazzino;
+-------+----+
| Sede  | ID |
+-------+----+
| SedeA |  1 |
| SedeA |  2 |
| SedeA |  3 |
| SedeB |  1 |
| SedeB |  2 |
| SedeB |  3 |
| SedeC |  1 |
| SedeC |  2 |
+-------+----+
\end{Verbatim}
Impostando l'{\tt AUTO\_INCREMENT} sull'attributo {\tt ID} si avrebbe invece:
\begin{Verbatim}
$ SELECT * FROM Magazzino;
+----+-------+
| ID | Sede  |
+----+-------+
|  1 | SedeA |
|  2 | SedeA |
|  3 | SedeA |
|  4 | SedeB |
|  5 | SedeB |
|  6 | SedeB |
|  7 | SedeC |
|  8 | SedeC |
+----+-------+
\end{Verbatim}
L'{\it engine} {\tt MyISAM} permette di impostare l'{\tt AUTO\_INCREMENT} su una colonna {\it secondaria}, ma
non permette i {\it vincoli di integrità referenziale} che sono fondamentali nel database di questo progetto. Dobbiamo
quindi utilizzare {\tt InnoDB} e {\it simulare} l'{\tt AUTO\_INCREMENT} mediante trigger.

\vspace{5pt}
Per brevità, gli {\tt INSERT INTO} che popolano le tabelle non sono riportati in questo documento (avrebbero
occupato oltre 100 pagine da soli!), ma sono comunque disponibili nello {\it script in allegato}.

\inputminted[%
              frame=leftline,           % bordo a sinistra
              linenos,                  % attiva numeri di linea
              stepnumber=5,             % solo multipli di 5 come numeri di linea
              tabsize=4,                % dimensione del tasto tab in spazi
              fontsize=\small]%         % dimensione del font (perché stia nella pagina bisogna assicurarsi che ogni riga di codice abbia al max 80 caratteri)
                              {mysql}{./sql/unipi_project_print.sql}
