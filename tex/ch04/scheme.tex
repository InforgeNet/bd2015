\section{Schema logico}
Di seguito è mostrato lo {\it schema logico} risultante dalla fase di progettazione logica.

Le {\it chiavi primarie} sono mostrate \underline{sottolineate} mentre gli {\it attributi opzionali}
sono rappresentati in \textit{corsivo}.

\begin{Verbatim}[commandchars=+\[\]]
SEDE(+underline[Nome], Città, CAP, Via, NumeroCivico)
MAGAZZINO(+underline[ID], +underline[Sede])
CONFEZIONE(+underline[CodiceLotto], Ingrediente, Peso, Prezzo, DataAcquisto,
    DataArrivo, Sede, Magazzino, +textit[Collocazione], +textit[Scadenza], +textit[Aspetto], Stato)
INGREDIENTE(+underline[Nome], Provenienza, TipoProduzione, Genere, Allergene)
CUCINA(+underline[Sede], +underline[Strumento], Quantità)
STRUMENTO(+underline[Nome])
FUNZIONE(+underline[Strumento], +underline[Nome])
MENU(+underline[ID], Sede, DataInizio, DataFine)
ELENCO(+underline[Menu], +underline[Ricetta], Novità)
RICETTA(+underline[Nome], Testo)
FASE(+underline[Ricetta], +underline[Numero], +textit[Ingrediente], +textit[Dose], +textit[Primario], +textit[Strumento], +textit[Testo], +textit[Durata])
SEQUENZAFASI(Ricetta, Fase, FasePrecedente)
PIATTO(+underline[ID], Comanda, Ricetta, Stato)
MODIFICA(+underline[Piatto], +underline[Variazione])
VARIAZIONE(+underline[ID], Ricetta, +textit[Nome], +textit[Account])
MODIFICAFASE(+underline[ID], +underline[Variazione], Ricetta, +textit[FaseVecchia], +textit[FaseNuova])
COMANDA(+underline[ID], Timestamp, Sede, +textit[Sala], +textit[Tavolo], +textit[Account], Stato)
TAVOLO(+underline[ID], +underline[Sala], +underline[Sede], Posti)
SALA(+underline[ID], +underline[Sede])
CONSEGNA(+underline[Comanda], Pony, Partenza, +textit[Arrivo], +textit[Ritorno])
PONY(+underline[ID], Sede, Ruote, Stato)
PRENOTAZIONE(+underline[ID], Sede, Data, Numero, +textit[Account], +textit[Nome], +textit[Telefono], Sala,
    +textit[Tavolo], +textit[Descrizione], +textit[Approvato])
ACCOUNT(+underline[Username], Email, Password, Nome, Cognome, Città, CAP, Via,
    NumeroCivico, Telefono, PuòPrenotare)
PROPOSTA(+underline[ID], Account, Nome, +textit[Procedimento])
COMPOSIZIONE(+underline[Proposta], +underline[Ingrediente])
GRADIMENTO(+underline[ID], Account, +textit[Proposta], +textit[Variazione], Punteggio)
RECENSIONE(+underline[ID], Account, Ricetta, Testo, Giudizio)
VALUTAZIONE(+underline[Account], +underline[Recensione], Veridicità, Accuratezza, Testo)
DOMANDA(+underline[Numero], +underline[Sede], Testo)
RISPOSTA(+underline[Numero], +underline[Domanda], +underline[Sede], Testo, Efficienza)
QUESTIONARIOSVOLTO(+underline[Recensione], +underline[Sede], +underline[Domanda], +underline[Risposta])
\end{Verbatim}
