\section{ModificaFase}\label{sec:modificafase}
La relazione {\tt Confezione} non rispetta la BCNF. Riportiamo di seguito tutte
le dipendenze funzionali non banali della relazione in questione:
\begin{funcdep}{ModificaFase}
    Variazione $\to$ Ricetta\\
    Variazione, ID $\to$ FaseVecchia, FaseNuova
\end{funcdep}
\noindent\texttt{Variazione}, da sé, non è infatti {\it superchiave} della relazione {\tt ModificaFase}.
Per poter normalizzare tale relazione è necessario spostare l'attributo {\tt Ricetta} nella
relazione {\tt Variazione}. Questo però ci impone anche di dover trovare un nuovo identificatore
per {\tt Fase} --- useremo quindi un campo {\tt ID} e toglieremo {\tt Numero} (superfluo in quanto l'ordine
delle fasi è dato dalla relazione {\tt SequenzaFasi}). Bisognerà quindi modificare le
relazioni coinvolte come segue:

\begin{Verbatim}[commandchars=+\[\]]
FASE(+underline[ID], Ricetta, +textit[Ingrediente], +textit[Dose], +textit[Primario], +textit[Strumento], +textit[Testo],
    +textit[Durata])
SEQUENZAFASI(+underline[Fase], +underline[FasePrecedente])
VARIAZIONE(+underline[ID], Ricetta, +textit[Nome], +textit[Account])
MODIFICAFASE(+underline[Variazione], +underline[ID], +textit[FaseVecchia], +textit[FaseNuova])
\end{Verbatim}
Anche i {\it vincoli di integrità referenziale} dovranno cambiare di conseguenza: dovrà
essere aggiunto il vincolo tra {\tt Ricetta} di {\tt Variazione} e {\tt Nome} di {\tt Ricetta}; dovranno
inoltre essere sistemati tutti i vincoli che si riferiscono alle tuple di {\tt Fase} in quanto
adesso vengono identificate dall'unico attributo {\tt ID}.

\vspace{10pt}
\noindent Si vede immediatamente che, così facendo, la BCNF è rispettata per tutte le relazioni modificate.
