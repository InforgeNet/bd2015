\section{QuestionarioSvolto}\label{sec:questionariosvolto}
La relazione {\tt QuestionarioSvolto} non rispetta la BCNF. Riportiamo di seguito tutte
le dipendenze funzionali non banali della relazione in questione:
\begin{funcdep}{QuestionarioSvolto}
    Recensione $\to$ Sede\\
    Recensione, Domanda $\to$ Risposta
\end{funcdep}
\noindent\texttt{Recensione}, da sé, non è infatti {\it superchiave} della relazione {\tt QuestionarioSvolto}. Nemmeno
({\tt Recensione}, {\tt Domanda}) è {\it superchiave} della relazione.
Per poter normalizzare tale relazione è necessario spostare l'attributo {\tt Sede} nella
relazione {\tt Recensione}. Questo però ci impone anche di dover trovare un nuovo identificatore
per le relazioni {\tt Risposta} e {\tt Domanda} --- useremo quindi un campo {\tt ID} per {\tt Domanda} e
toglieremo {\tt Numero} (superfluo in quanto l'ordine delle domande può essere dato dalla sequenza
degli {\tt ID}). Inoltre possiamo (anche se non necessario per la normalizzazione)
togliere {\tt Risposta} dall'identificatore di {\tt QuestionarioSvolto} (è
possibile in quanto, per ogni recensione, l'utente può dare una sola risposta ad ogni
domanda --- come specificato dalla business rule \ref{br.surveyanswers}). Bisognerà
quindi modificare le relazioni coinvolte come segue:

\begin{Verbatim}[commandchars=+\[\]]
RECENSIONE(+underline[ID], Account, Sede, Ricetta, Testo, Giudizio)
DOMANDA(+underline[ID], Sede, Testo)
RISPOSTA(+underline[Domanda], +underline[Numero], Testo, Efficienza)
QUESTIONARIOSVOLTO(+underline[Recensione], +underline[Domanda], Risposta)
\end{Verbatim}
Anche i {\it vincoli di integrità referenziale} dovranno cambiare di conseguenza: dovrà
essere aggiunto il vincolo tra {\tt Sede} di {\tt Recensione} e {\tt Nome} di {\tt Sede}; dovranno
inoltre essere sistemati il vincolo tra {\tt QuestionarioSvolto} e {\tt Risposta} in quanto le tuple di quest'ultima
adesso vengono identificate dai soli attributi {\tt Domanda} e {\tt Numero}. Ovviamente deve essere
sistemato anche il vincolo tra {\tt Risposta} e {\tt Domanda} in quanto le tuple di quest'ultima adesso
vengono identificato dall'unico attributo {\tt ID}.

\vspace{10pt}
\noindent Si vede immediatamente che, così facendo, la BCNF è rispettata per tutte le relazioni modificate.
