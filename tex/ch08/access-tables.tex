\section{Tavole degli accessi}\label{sec:accesstables}
Vediamo le {\it tavole degli accessi} delle operazioni individuate nel Capitolo~\vref{ch:operations}.
\vspace{10pt}

Per la prima operazione si nota facilmente che sono necessarie cinque letture su {\tt Piatto}, in
quanto ogni comanda ha in media 5 piatti, e una su {\tt Comanda}:
{\tabulinesep=3pt
\begin{longtabu} to \linewidth {|X[2,c,m]|X[c,m]|X[c,m]|}
\hline\rowfont\bfseries
\multicolumn{3}{|c|}{\large Operazione 1}
\\\hline\hline\hline\hline
\textbf{Tabella}                        & \textbf{Accessi}      & \textbf{Tipo}
\\ \hline \hline \hline % ------------------------------------------------------
\endhead
Piatto                                  & \(5\)                 & L
    \\ \hline % ----------------------------------------------------------------
Comanda                                 & \(1\)                 & L
    \\ \hline\hline\hline % ----------------------------------------------------
\multicolumn{3}{|l|}{\textbf{Costo totale:} \(5 + 1 = 6 \times 1\,500 = \textbf{9\,000}/giorno\)}
    \\ \hline % ----------------------------------------------------------------
\end{longtabu}}

Per la seconda operazione il {\it trigger} {\tt assegna\_pony} effettua una chiamata a {\tt StatoComanda} la
quale effettua cinque letture su {\tt Piatto} e una lettura su {\tt Comanda}, poi altre cinque letture su {\tt Piatto} (per
contare il numero di piatti), quattro letture su {\tt Pony}, in quanto ogni sede ha in media 4 pony, e una scrittura
su {\tt Consegna} per piazzare la nuova consegna. Quest'ultima scrittura causerà a sua volta l'esecuzione di un trigger
che provedderà ad aggiornare lo stato del pony su {\tt 'occupato'} --- quindi effettua una scrittura anche su {\tt Pony}:
{\tabulinesep=3pt
\begin{longtabu} to \linewidth {|X[2,c,m]|X[c,m]|X[c,m]|}
\hline\rowfont\bfseries
\multicolumn{3}{|c|}{\large Operazione 2}
\\\hline\hline\hline\hline
\textbf{Tabella}                        & \textbf{Accessi}      & \textbf{Tipo}
\\ \hline \hline \hline % ------------------------------------------------------
\endhead
Comanda                                 & \(1\)                 & L
    \\ \hline % ----------------------------------------------------------------
Piatto                                  & \(10\)                & L
    \\ \hline % ----------------------------------------------------------------
Pony                                    & \(4\)                 & L
    \\ \hline % ----------------------------------------------------------------
Consegna                                & \(1\)                 & S
    \\ \hline % ----------------------------------------------------------------
Pony                                    & \(1\)                 & S
    \\ \hline\hline\hline % ----------------------------------------------------
\multicolumn{3}{|l|}{\textbf{Costo totale:} \(1 + 10 + 4 + 1 \times 2 + 1 \times 2 = 19 \times 125 = \textbf{2\,375}/giorno\)}
    \\ \hline % ----------------------------------------------------------------
\end{longtabu}}

\clearpage
La terza operazione è una scrittura su {\tt Comanda}:
{\tabulinesep=3pt
\begin{longtabu} to \linewidth {|X[2,c,m]|X[c,m]|X[c,m]|}
\hline\rowfont\bfseries
\multicolumn{3}{|c|}{\large Operazione 3}
\\\hline\hline\hline\hline
\textbf{Tabella}                        & \textbf{Accessi}      & \textbf{Tipo}
\\ \hline \hline \hline % ------------------------------------------------------
\endhead
Comanda                                 & \(1\)                 & S
    \\ \hline\hline\hline % ----------------------------------------------------
\multicolumn{3}{|l|}{\textbf{Costo totale:} \(1 \times 2 = 2 \times 675 = \textbf{1\,350}/giorno\)}
    \\ \hline % ----------------------------------------------------------------
\end{longtabu}}

La quarta operazione è una scrittura su {\tt Piatto}:
{\tabulinesep=3pt
\begin{longtabu} to \linewidth {|X[2,c,m]|X[c,m]|X[c,m]|}
\hline\rowfont\bfseries
\multicolumn{3}{|c|}{\large Operazione 4}
\\\hline\hline\hline\hline
\textbf{Tabella}                        & \textbf{Accessi}      & \textbf{Tipo}
\\ \hline \hline \hline % ------------------------------------------------------
\endhead
Piatto                                  & \(1\)                 & S
    \\ \hline\hline\hline % ----------------------------------------------------
\multicolumn{3}{|l|}{\textbf{Costo totale:} \(1 \times 2 = 2 \times 3\,375 = \textbf{6\,750}/giorno\)}
    \\ \hline % ----------------------------------------------------------------
\end{longtabu}}

Per la quinta operazione la funzione {\tt IngredientiDisponibili} effettua 15 letture
su {\tt Prenotazione}, in quanto una sede riceve in media 15 prenotazioni al giorno, cinque letture
su {\tt Clienti\_Log}, ipotizzando che la catena di ristorazione sia aperta da 5 anni, una
lettura su {\tt MV\_OrdiniRicetta}, sette letture su {\tt Fase}, ipotizzando che una ricetta
sia composta in media da 7 fasi che aggiungono ingredienti, due letture su {\tt Confezione}, ipotizzando
che in media un magazzino contiene una sola confezione di ogni ingrediente e una sede possiede due magazzini.
{\tabulinesep=3pt
\begin{longtabu} to \linewidth {|X[2,c,m]|X[c,m]|X[c,m]|}
\hline\rowfont\bfseries
\multicolumn{3}{|c|}{\large Operazione 5}
\\\hline\hline\hline\hline
\textbf{Tabella}                        & \textbf{Accessi}      & \textbf{Tipo}
\\ \hline \hline \hline % ------------------------------------------------------
\endhead
Prenotazione                            & \(15\)                & L
    \\ \hline % ----------------------------------------------------------------
Clienti\_Log                            & \(5\)                 & L
    \\ \hline % ----------------------------------------------------------------
MV\_OrdiniRicetta                       & \(1\)                 & L
    \\ \hline % ----------------------------------------------------------------
Fase                                    & \(7\)                 & L
    \\ \hline % ----------------------------------------------------------------
Confezione                              & \(2\)                 & L
    \\ \hline\hline\hline % ----------------------------------------------------
\multicolumn{3}{|l|}{\textbf{Costo totale:} \(15 + 5 + 1 + 7 + 2 = 30 \times 500 = \textbf{15\,000}/giorno\)}
    \\ \hline % ----------------------------------------------------------------
\end{longtabu}}

La sesta operazione è una scrittura su {\tt Prenotazione}:
{\tabulinesep=3pt
\begin{longtabu} to \linewidth {|X[2,c,m]|X[c,m]|X[c,m]|}
\hline\rowfont\bfseries
\multicolumn{3}{|c|}{\large Operazione 6}
\\\hline\hline\hline\hline
\textbf{Tabella}                        & \textbf{Accessi}      & \textbf{Tipo}
\\ \hline \hline \hline % ------------------------------------------------------
\endhead
Prenotazione                            & \(1\)                 & S
    \\ \hline\hline\hline % ----------------------------------------------------
\multicolumn{3}{|l|}{\textbf{Costo totale:} \(1 \times 2 = 2 \times 375 = \textbf{750}/giorno\)}
    \\ \hline % ----------------------------------------------------------------
\end{longtabu}}

Per la settima operazione è necessario, per poter stilare la classifica, leggere
tutte le istanze di {\tt Recensione} e tutte quelle di {\tt Valutazione}:
{\tabulinesep=3pt
\begin{longtabu} to \linewidth {|X[2,c,m]|X[c,m]|X[c,m]|}
\hline\rowfont\bfseries
\multicolumn{3}{|c|}{\large Operazione 7}
\\\hline\hline\hline\hline
\textbf{Tabella}                        & \textbf{Accessi}      & \textbf{Tipo}
\\ \hline \hline \hline % ------------------------------------------------------
\endhead
Recensione                              & \(2\,000\)            & L
    \\ \hline % ----------------------------------------------------------------
Valutazione                             & \(4\,000\)            & L
    \\ \hline\hline\hline % ----------------------------------------------------
\multicolumn{3}{|l|}{\textbf{Costo totale:} \(2\,000 + 4\,000 = 6\,000 \times 100 = \textbf{60\,000}/giorno\)}
    \\ \hline % ----------------------------------------------------------------
\end{longtabu}}

L'ottava operazione è una scrittura su {\tt Valutazione}:
{\tabulinesep=3pt
\begin{longtabu} to \linewidth {|X[2,c,m]|X[c,m]|X[c,m]|}
\hline\rowfont\bfseries
\multicolumn{3}{|c|}{\large Operazione 8}
\\\hline\hline\hline\hline
\textbf{Tabella}                        & \textbf{Accessi}      & \textbf{Tipo}
\\ \hline \hline \hline % ------------------------------------------------------
\endhead
Valutazione                             & \(1\)                 & S
    \\ \hline\hline\hline % ----------------------------------------------------
\multicolumn{3}{|l|}{\textbf{Costo totale:} \(1 \times 2 = 2 \times 3 = \textbf{6}/giorno\)}
    \\ \hline % ----------------------------------------------------------------
\end{longtabu}}
