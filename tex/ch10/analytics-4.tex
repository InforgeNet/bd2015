\section{Analisi dei consumi e degli sprechi}
La seguente procedura calcola, per ogni sede e per ogni ingrediente, la differenza
tra la materia prima scaricata dai magazzini della sede e quella effettivamente impiegata
nella produzione dei piatti ordinati dai clienti, in un certo periodo di tempo.

Per far ciò la procedura itera per tutte le sedi e per tutti gli ingredienti e prende
dalla tabella {\tt Scarichi\_Log} la quantità totale di ingrediente scaricata dai magazzini
della sede. Poi fa uso di una query molto complessa per calcolare la quantità di ingrediente
usata nella produzione di tutti i piatti ordinati che richiedono tale ingrediente.

La query prende tutti i piatti prodotti dalla sede nel periodo di interesse specificato. Per ogni
piatto prende tutte le fasi di preparazione che prevendono l'aggiunta dell'ingrediente preso in
analisi. Per ottenere le fasi di preparazione effettivamente impiegate nella produzione del piatto, la
query deve anche considerare le variazioni: la prima delle due condizioni nel {\tt WHERE} che
coinvolge {\tt F.ID} controlla che la fase non sia presente come {\tt FaseVecchia} all'interno di
una qualche {\tt ModificaFase} di una qualche {\tt Variazione} scelta dal cliente per quel
piatto; la seconda controlla che la fase non sia presente come {\tt FaseNuova} all'interno di
una qualche {\tt ModificaFase} di una qualche {\tt Variazione} {\bf non} scelta dal cliente
per quel piatto. Il {\it result-set} risultate contiene tutte le fasi di aggiunta dell'ingrediente
preso in analisi con tutte le informazioni sul piatto, la comanda e la ricetta alla quale si
riferiscono. Quindi le stesse fasi che aggiungono tale ingrediente sono ripetute per il numero
di piatti prodotti per ogni ricetta. Sommando tutte le dosi si ottiene la quantità totale di
ingrediente impiegato nella produzione dei piatti.

La tabela {\tt Report\_TakeAway} riporta, per ogni sede e per ogni ingrediente, la quantità
di materia prima sprecata.

\inputminted[%
              frame=leftline,           % bordo a sinistra
              linenos,                  % attiva numeri di linea
              stepnumber=5,             % solo multipli di 5 come numeri di linea
              tabsize=4,                % dimensione del tasto tab in spazi
              fontsize=\small]%         % dimensione del font (perché stia nella pagina bisogna assicurarsi che ogni riga di codice abbia al max 80 caratteri)
                              {mysql}{./sql/analytics-4.sql}
